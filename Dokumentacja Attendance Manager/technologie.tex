\newpage\section{Wybór technologii informatycznych} \label{sec:technologie}
\subsection{ASP.NET Core Web API}
Do stworzenia interfejsu webowego odpowiadającego za pośrednictwo w komunikacji pomiędzy aplikacja kliencką, bazą danych oraz panelem administracyjnym w przeglądarce Internetowej zostało wykorzystane środowisko ASP.NET Core Web API. Jest to nowoczesne narzędzie o otwartym kodzie źródłowym stworzone przez firmę Microsoft. Umożliwia ono tworzenie wydajnej części serwerowej systemów na każdej platformie.

\subsection{MS SQL Server 2016}
Do przechowywania danych został wykorzystany serwer bazodanowy Microsoft SQL Server 2016 wraz z relacyjną bazą danych. Jest to stabilne i sprawdzone rozwiązanie, które doskonale poradzi sobie z zdaniem przechowywania dużej ilosci danych o stałej strukturze, co wpisuje się w specyfikację systemu.

\subsection{Microsoft Azure}
Całość rozwiązania po stronie serwera została umieszczona w chmurze hybrydowej Microsoft Azure. Umożliwia to wygodne i efektywne zarządzanie i monitorowanie pracą systemu bez potrzeby tworzenia infrastruktury, np. maszyn wirtualnych na fizycznych komputerach.

\subsection{Front End}
\subsection{.NET applikacja}
\subsection{Latex do dokumentacji}
\subsection{Format JSON}
(ang. \textit{JavaScript Object Notation}) \cite{json2017}
\subsection{Środowisko programistyczne Visual Studio}
Do prac nad elementami systemu stworzonymi w technologiach Microsoft .NET wykorzystane zostało środowisko programistyczne Microsoft Visual Studio. Jest to kompleksowe rozwiązanie umożliwiające programistom tworzenie i zarządzanie dużymi rozwiązaniami.