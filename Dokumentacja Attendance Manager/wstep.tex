\newpage\section{Wstęp}\label{sec:wstep}
\subsection{Cel i zakres pracy}
% Odpowiedź na pytanie: Do czego system służy?
% Przedstawić zadania szczegółowe z karty tematu 

System ma za zadanie monitorować obecności studentów na zajęciach laboratoryjnych, wykorzystując istniejącą infrastrukturę informatyczną, tj. czytniki kart inteligentnych zainstalowane w komputerach oraz legitymacje studenckie.
% Priorytet: Wysoki, Średni, Niski
\begin{table}[!ht]
\centering
    \begin{tabular}{|c|p{6cm}|c|c|}
        \hline
        \textit{Lp.} & \textit{Opis funkcjonalności} & \textit{Dostępność}  & \textit{Priorytet} \\ \hline
        1. & Odczytanie danych z legitymacji & - & Wysoki \\ \hline
        2. & Ręczne wprowadzenie danych o obecności& a & Średni \\ \hline
        3. & Dostęp do panelu administracyjnego prez web serwis & a & Wysoki \\ \hline
        4. & Podgląd istniejących w systemie wydarzeń & a & Wysoki \\ \hline
        5. & Planowanie wydarzeń jednorazowych & a & Wysoki \\ \hline
        6. & Planowanie wydarzeń cyklicznych & a & Wysoki \\ \hline
        7. & Ustalanie list dopuszczonych do wydarzenia uczestnikow & a & Średni \\ \hline
    \end{tabular}
    \caption{Funkcjonalność systemu \NazwaSys (a -- administrator)}
    \label{table:tab1}
\end{table}

\subsection{Podział pracy}

\begin{table}
\centering
\caption{Podział pracy}
    \begin{tabular}{|p{4cm}|p{5cm}|}
        \hline
        \textit{Osoba odpowiedzialna} & \textit{Zadanie} \\ \hline
        Krzysztof Adamczak & Front end, Web serwis \\ \hline
        Wojciech Agaciński & Web serwis, Baza danych  \\ \hline
        Jakub Piotr & Aplikacja kliencka  \\  \hline
    \end{tabular} 
    \label{table:tab2}
\end{table}