% 
\newpage\section{Opis dziedziny przedmiotowej pracy}\label{sec:dziedzina}


\subsection{Pojęcia i definicje}
\begin{itemize}
  \item Karta elektroniczna - uniwersalny nośnik danych w postaci karty wykonanej z plastiku z umieszczonym na niej (lub wewnątrz niej) jednym lub kilkoma układami scalonymi (chip), które pozwalają na ochronę procesu logowania użytkownika, kontrolę dostępu i zawartych na niej danych. Może być odczytywana za pomocą urządzeń automatycznych, np. przy zawieraniu i rozliczaniu transakcji finansowych oraz w kasach cyfrowych. Karty elektroniczne mają rozmiar zgodny z formatem ID-1 (85,60 × 53,98 mm) określony normą ISO/IEC 7810 tak jak tradycyjne karty kredytowe z paskiem magnetycznym. Często posiadają również taki pasek i mogą być odczytywane w urządzeniach nie obsługujących kart elektronicznych.
  \item Elektroniczna Legitymacja Studencka (ELS) - rodzaj legitymacji, nowa postać tradycyjnej papierowej legitymacji studenckiej. Została wprowadzona przepisami Rozporządzenia Ministra Nauki i Szkolnictwa Wyższego z dnia 2 listopada 2006 w sprawie dokumentacji przebiegu studiów. Elektroniczna Legitymacja Studencka ma formę karty z tworzywa sztucznego z wbudowanym procesorem, a jej wzór graficzny jest jednolity dla wszystkich uczelni i został określony w ww. rozporządzeniu. Legitymacja wyposażona jest w chip stykowy oraz może być wyposażona w bezstykowy interfejs Mifare. Według rozporządzenia jest ona dokumentem poświadczającym status studenta. Dodatkowo może być ona wykorzystywana jako karta biblioteczna, karta dostępu do laboratoriów lub bilet komunikacji miejskiej.
\end{itemize}
\subsection{Stan wiedzy}
Powszechnie dostępne czytniki kart elektronicznych, wśród których można wymienić te wbudowane w komputery laboratoryjne na terenie Politechniki Poznańskiej oraz moduły NFC w nowoczesnych smartfonach - pozwalają uzyskać dostep do danych zapisanych na Elektronicznych Legitymacjach Studenckich. Te dane pozwalają nie tylko zidentyfikować posiadacza karty, ale także określić jego uprawnienia jak i przynależność wewnątrz jednostek uczelnianych.

\subsection{Dyskusja} 
Wykorzystanie Elektronicznych Legitymacji Studenckich do rejestracji obecności może budzić wiele obaw odnosnie bezpieczeństwa informacji. Legitymacje zawierają dane osobowe uważane za poufne, m. in PESEL, dlatego też wrażliwe dane odczytane przez sytem powinny być odpowiednio zabezpieczone przed nieuprawnionym dostępem, a najlepiej niedostępne dla administratorów oprócz przypadków w których te dane są niezbędne.