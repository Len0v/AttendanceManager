\newpage\section{Implementacja} \label{sec:implementacja}
\subsection{Wzorce projektowe i strukturalne}
\begin{itemize}  
    \item Model View Controller - wzorzec strukturalny wspierający zasadę Separation of Concerns - podział części serwerowej systemu na częsci odpowiedzialne za widoki, logikę biznesową oraz obsługę zapytań HTTP
    \item Inversion of Control - wzorzec strukturalny odpowiedzialny za zapewnienie luźnego powiązania pomiędzy klasami i likwidowanie zależności poprzez wykorzystanie Wstrzykiwania Zależności.
    \item Unit of Work - wzorzec projektowy zapewniający operowanie na jednej jednostce transakcji bazodanowych w obrębie zapytania do serwisu
    \item Repository Pattern - wzorzec projektowy zapewniający spójny format operacji CRUD na różnych strukturach danych 
\end{itemize}
\subsection{Model struktur danych}
Struktury danych odzwierciedlone w strukturach bazodanowych składają się z klas według modelu Plain Old CLR Object
\subsection{Model klas}
Podstawowe klasy występujące w systemie
\begin{itemize}
    \item Attendee - uczestnik wydarzenia
    \item Course - przedmiot dydaktyczny
    \item CourseType - rodzaj zajęć
    \item CourseUnit - jednostka dydaktyczna obejmująca przedmiot oraz rodzaj zajęć
    \item Event - wydarzenie
    \item Lecturer - wykładowca, prowadzący zajęcia
    \item Room - pomieszczenie
    \item TimeSlot - przedział czasowy
\end{itemize}
\subsection{Wnioski}




 
